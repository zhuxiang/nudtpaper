\chapter{基于卷积神经网络的文本分类研究}
近年来,深度学习(Deep Learning)在计算机视觉\upcite{krizhevsky2012imagenet}以及语音识别\upcite{graves2013speech}等方面取得了显著的成果。在自然语言处理领域,深度学习方法对于语义理解以及文本表示等方面也进行了技术的革新,相关的工作包含基于神经语言模型进行词向量表示的学习\upcite{bengio2003neural,yih2011learning,mikolov2013distributed}以及基于已学习的词向量模型进行文本分类\upcite{collobert2011natural}等。在社交网络信息传播分析中,文本分类的研究一直是社交网络数据分析与挖掘的基础和热点。社交网络中海量的用户生成数据(User-Generated Content)提供了话题种类繁多的信息。如何实现社交网络中的文本自动分类是一个极具挑战的任务。

由于社交网络中文本的数据海量性、话题多样性以及数据稀疏性等特点,传统的文本分类技术的效率将变低,无法很好的解决文本自动分类任务。例如词袋模型(Bag of Words Model)的词向量维度将随着社交网络中的词组数量增加而增加,词向量的稀疏性将给语义模型的训练带来精确性上的降低,并且使得模型训练的时间开销增加。卷积神经网络(Convolutional Neural Networks)将卷积核应用到局部特征上来进行语义的处理\upcite{lecun1998gradient}。该技术首先是用于计算机视觉领域,同时CNN模型近年来在自然语言处理领域上也取得了很好的效果,例如语义分析\upcite{yih2014semantic}、查询检索\upcite{shen2014learning}、语句建模\upcite{kalchbrenner2014convolutional}以及其他的自然语言处理任务\upcite{collobert2011natural}。本章针对传统文本分类方法的不足,利用深度学习的方法对文本分类问题进行处理,对文本中语句的重要性进行排序,提取核心语句集合作为文本的语义表示,训练设计的CNN模型参数,实现文本的自动分类。

本章主要的工作可以总结如下。首先,结合卷积神经网络的特性,本章提出了一个面向社交网络的文本自动分类框架。其次,本章提出了一个核心语句提取的算法,在保证文本语义的同时降低了计算的复杂度,而且保留了文本语句中的词序。然后,本章利用外部语料库训练好的词向量模型对文本进行表示,将文本转换成一个语义矩阵,利用社交网络中标注好的预料对CNN模型参数进行训练,实现文本自动分类。最后,本章在真实数据集上进行了实验,与传统的方法进行对比,验证了算法的有效性。

本章的内容组织如下:第\ref{sec3:motivation}节介绍了研究动机,讨论了传统的文本分类方法的不足以及深度学习方法对于文本分类的帮助。第\ref{sec3:definition}节介绍了相关定义,对本章中相关概念和所提出的问题进行了符号化的定义。第\ref{sec3:method}介绍了方法描述,详细地阐述了本章所提出的框架以及相关算法的详细过程。第\ref{sec3:experiment}节进行了实验分析,设计了一系列的实验,验证了本章所提出的方法,并对实验结果进行了分析。最后,第\ref{sec3:conclusion}对本章的内容进行了总结。
\section{研究动机}
\label{sec3:motivation}

\section{相关定义}
\label{sec3:definition}

\section{方法描述}
\label{sec3:method}
在本节中,我们借鉴了Kim的思想\upcite{kim2014convolutional},并且做出了改进,实现了一个基于卷积神经网络的文本分类器。针对社交网络中的长文本信息(例如新闻、长微博等)的话题分类问题,我们设计了一个基于深度学习的框架来解决文本的分类问题。详细的算法将在本节中进行阐述。首先,我们介绍文本摘要算法,该算法用于长文本信息中的关键语句的提取。相对于Mihalcea所提出的基于图排序算法的语句提取算法,我们所提出的文本摘要算法更加的简洁。算法基于词语的tf-idf值来计算语句的重要性,该算法相对耗时短,更加适用于社交网络中长文本的关键语句提取。其次,我们选取top-\textit{k}个语句来作为长文本信息的摘要。这个过程减少了长文本信息的规模,并且消除了其中的大量的噪声语句。然后,基于外部语料库(例如维基百科等),我们建立了一个词向量模型(Word Vector Model)。该模型将关键语句中的词语转换成向量。在这一步骤中,我们仍然保留了语句中的词序顺序关系,即上下文关系。最终,我们利用标注好的数据集来训练所设计的三层卷积神经网络。各个步骤的详细描述在下面的各个小节中进行描述。

\subsection{文本摘要提取}
\label{subsec3:abstactExtract}
对于文本摘要提取,其核心的思想是对文本信息中的语句进行排序,遴选出其中最能代表文本主题的语句。这个步骤在整个文本分类中是比较重要的,它能够降低文本信息的规模,消除噪声语句,减少文本分类器的训练时间。社交网络中的长文本信息包含的语句可能会比较多,为了保证文本分类的效率,我们利用文本摘要算法来降低文本中语句的规模。在本小节中,基于词语的tf-idf值,我们实现了一种文本摘要算法来提取长本文信息中的关键语句。语句按照其中词语的tf-idf值进行排序,排序靠前的top-\textit{k}个语句被选择作为该信息的摘要,用来代表信息的核心语义。

\section{实验分析}
\label{sec3:experiment}

\section{本章小结}
\label{sec3:conclusion}