\chapter{社交网络中传播效率最大化}
\label{3chap:main}
\textbf{影响力最大化}(\textit{Influence Maximization})在许多的社交网络应用中扮演着举足轻重的角色,例如市场营销、商业活动以及竞选活动等等。研究信息传播模型以及机理的一个典型目的便是市场营销。在现实社会中,人或者事物都是通过某种关系来连接,形成了一个巨大的网络。因此,信息传播可以选择一小部分节点激活做为种子节点集合,然后通过信息的传递,在整个网络中产生一个大范围的影响。从技术方面来说,影响力最大化问题是在给定网络和传播模型的条件下,研究如何选择种子节点集合使得传播的影响范围最大化。影响力最大化问题由于其的问题真实、应用性广的特点,吸引了许许多多的研究,包括信息传播模型的研究和计算影响力最大化的方法。但是,仍然有着一些需求在实际应用中得不到满足。

在信息传播过程中,一个被激活的节点将会尝试在下一个时刻激活它的邻居节点。所以,除去种子节点外,每一个被激活的节点在被激活之前都会有一个时间延迟,我们称之为\textbf{传播时延}(\textit{propagation time delay})。如果一个节点在信息传播结束时,仍然未被激活,则该节点的传播时延可以看作无穷大。在传统的影响力最大化问题中,传播时延没有被考虑在内,而研究整个网络的传播时延是非常有意义的,它可以度量选择的种子节点集合的传播效率。出于这种需求,本章提出了一个新的问题,传播效率最大化问题。该问题将传播时延考虑在内,在给定网络和传播模型的条件下,研究如何选择种子节点集合使得传播效率最大化。传播效率最大化问题与影响力最大化问题虽然相似,但是两个问题的侧重点不尽相同。传统的影响力最大化问题研究的是如何使得传播的范围最广,不考虑节点的传播时延。而本章提出的传播效率最大化问题将传播时延考虑在内,研究如何使得传播效率最大化。

本章主要的工作可以总结如下。首先,基于传统的影响力最大化问题,我们将传播时延考虑在内了来探究传播效率,提出了传播效率最大化问题,研究给定网络和传播模型的条件下,研究如何选择种子节点集合使得传播效率最大化。其次,本章证明了传播效率最大化问题是一个NP-hard问题,而且在独立级联模型下计算传播效率的过程是一个\#P-hard的问题。然后,我们证明了传播效率函数在独立级联模型下是\textbf{子模}(\textit{submodular})的。最后本章设计了三个算法来解决提出的传播效率最大化问题,并且在真实数据集上验证了这些算法。实验结果展示了所提出的算法的性能。

本章的内容组织如下:第\ref{3sec:motivation}节介绍了研究动机,讨论了传统的影响力最大化问题的不足和考虑了传播时延的传播效率最大化问题的意义。第\ref{3sec:definition}节介绍了相关定义,对本章中相关概念和所提出的问题进行了符号化的定义。第\ref{3sec:method}介绍了方法描述,详细地阐述了本章解决问题的方法以及相关证明过程。第\ref{3sec:experiment}节进行了实验分析,设计了一系列的实验,验证了本章所提出的方法,并对实验结果进行了分析。最后,第\ref{3sec:conclusion}对本章的内容进行了总结。

\section{研究动机}
\label{3sec:motivation}
随着社交网络的兴盛发展,信息传播的速率变得越来越快。在传统的媒体环境下,一条消息需要通过较长时间才能传播到一定的范围。在社交网络中,个人通过关系(例如关注、好友等)连接形成网络,信息在网络中通过转发、复制等行为进行传播,信息传播的速率极大的加快了。在传统的媒体环境下,个人都是通过单一信息源(例如新闻、论坛等)来接收信息。而在社交网络环境下,个人既可以是信息的接收者也可以是信息的发布者,个人可以接收到在网络中与之相连的个人推送的信息。与此同时,信息通过个人构成的网络进行传播。影响力最大化问题是信息传播中的一个典型问题,是在给定网络和传播模型的条件下,研究如何选择种子节点集合使得传播的影响范围最大化。市场营销是研究信息传播模型以及机理的一个典型驱动。市场营销的过程可以简述如下,一个公司希望在社交网络中通过“口碑效应”推动一款新产品或者一种新理念。一种有成本效益的方法是寻找到整个网络中有影响力的个人,然后投入资源来使他们接受这款产品,例如赠送样品、免费试用、优惠折扣等。这样的举措是希望这些有影响力的个人在接受新产品或者新理念后,能够驱使社交网络中的其他个人也来接受新产品或者新理念,然后在社交网络中产生一个大的级联效应,从而使得更多的个体接受新产品或者新理念。为了达到市场营销这一目的,我们需要对两个重要的问题进行研究:(1)如何对网络中的信息传播过程进行建模,包括模型参数的学习等;(2)如何在给定的传播模型的条件下,设计一个有效的方法来寻找能够最大化影响力的节点集合。本章着重对第二个问题进行研究。

影响力最大化问题作为市场营销的一种算法技术首先被Domingos和Richardson\upcite{domingos2001mining}所提出,该问题基于马尔科夫随机场的概率框架。而后,Kempe等人\upcite{kempe2003maximizing}首次将影响力最大化问题形式化成为一个离散的随机优化问题。信息传播的过程可以简述如下,在一个给定的网络中,选择部分节点做为种子集合,这些种子节点将会按照一定的规则去激活它们的邻居节点。被激活的节点在下一时刻将拥有能力去激活它们的邻居节点,这个过程将一直持续到没有新的节点可以被激活,整个信息传播的过程才会停止。影响力最大化问题是在给定网络和传播模型的条件下,研究如何选择种子节点集合使得传播的影响范围最大化,即激活的节点数目最大化。从上述信息传播过程的描述中,我们可以得知,一个被激活的节点会在被激活的下一时刻尝试激活它的邻居节点。因此,除去种子节点外,网络中的节点在被激活前都会有一个时间延迟,我们称之为传播时延。如果一个节点在信息传播结束时,仍然未被激活,则该节点的传播时延可以看作无穷大。而影响力最大化问题仅仅考虑了传播范围,忽略了节点的传播时延。在真实的场景中,例如市场营销、商业活动、竞选活动等,传播时延在信息传播中是一个非常重要的因素。为了让其他人接受自己的新产品或者新理念,人们总是希望能够尽快地将消息传播到群体中。我们以\textbf{传播效率}(\textit{influence efficiency})来表示传播时延的倒数,传播时延小,则传播效率大。如果节点在信息传播结束时仍然未被激活,那么该节点的传播效率则为0。以上是针对单个节点的传播时延和传播效率的分析,下面我们对整个网络的传播时延和传播效率进行分析。在给定一个传播网络和初始的种子节点集合的情况下,如果整个网络中所有节点的传播效率高,这就意味着在信息传播过程中,网络中的节点将被迅速地激活。我们设想如下,针对传统的影响力最大化问题有两种选择种子节点集合的策略,它们有着相同的传播范围,即能够在信息传播过程结束时能够激活相同的节点数目。但是,这两种策略中,网络中的传播时延可能是不同的,即不同策略在同一网络中的传播效率是不同的,而这一问题在传统的影响力最大化问题中是没有讨论的。

\section{相关定义}
\label{3sec:definition}
在本节中,第\ref{3subsec:model}节首先对\textbf{独立级联模型}(\textit{Independent Cascade Model})以及在独立级联模型下的影响力最大化问题进行回顾,然后介绍了几种解决影响力最大化问题的方法,并且对这些算法进行分析,包括影响力函数期望的单调性和子模性等。其次,第\ref{3subsec:efficiency}节提出了\textbf{传播效率最大化}(\textit{Influence Efficiency Maximization})问题,该问题是基于传统的影响力最大化问题,将传播时延考虑在内,研究如何使得整个网络的传播效率最大化。同时,第\ref{3subsec:efficiency}节对传播效率最大化问题进行了形式化的描述,分析了传播效率最大化和影响力最大化问题的区别。

为了便于参照,表\ref{3tab:notation}中列出了频繁使用的符号。

\begin{table}[ht]
\centering
\caption{常用符号列表}
\begin{tabular}{|p{2cm}|p{10cm}|}
\hline
\textbf{符号} & \textbf{描述} \\
\hline
$\mathcal{G}=\left(\mathcal{V},\mathcal{E}\right)$ & $\mathcal{G}$是社交网络构成的图,$\mathcal{V}$是节点集合,$\mathcal{E}$是边的集合\\
\hline
$\mathcal{H}=\left(\mathcal{V},\mathcal{Z}\right)$ & $\mathcal{H}$是基于图$\mathcal{G}$生成的超图(参见\ref{alg:res}),$\mathcal{V}$是节点集合,$\mathcal{Z}$是超边集合\\
\hline
$n$ & $\mathcal{G}$或者$\mathcal{H}$的节点的数目 \\
\hline
$m$ & $\mathcal{G}$的边的数目 \\
\hline
$k$ & 种子节点集合的大小 \\
\hline
$p^\mathcal{G}_{u,v}$ & 节点$u$激活节点$v$的概率 \\
\hline
$I\left(S\right)$ & 种子节点集合$S$的影响力 \\
\hline
$RR\left(v\right)$ & 节点$v$的反向可达集合 (参见定义\ref{def:rrSet}) \\
\hline
$e_{u,v}$ & 节点$u$到节点$v$的传播效率(参见\ref{eq:efficiency}) \\
\hline
$T\left(S\right)$ & 种子节点集合$S$在图$\mathcal{G}$中的传播效率(参见\ref{eq:influenceEfficiency})\\
\hline
$T'\left(S\right)$ & 种子节点集合$S$在超图$\mathcal{H}$中的传播效率(参见\ref{alg:res})\\
\hline
\end{tabular}
\label{3tab:notation}
\end{table}

\subsection{传播模型以及影响力最大化问题}
\label{3subsec:model}
本章采用一种广泛采用的信息传播模型,独立级联模型,进行传播影响的研究。在该模型下,一个社交网络可被建模表示为一个有向图$\mathcal{G}=\left(\mathcal{V},\mathcal{E}\right)$,其中$\mathcal{V}$代表网络中的个体,$\mathcal{E}$表示个体之间的社会关系。此外,图中的每一条边$\left(u, v\right) \in \mathcal{E}$上都关联着一个传播概率$p^\mathcal{G}_{u,v}$,表示着节点$u$到节点$v$的影响力度。如果传播概率$p^\mathcal{G}_{u,v}$越大,则节点$u$更加可能激活节点$v$。如果图$\mathcal{G}$与上下文无关,本章则用$p_{u,v}$来表示节点$u$到节点$v$的传播概率。

独立级联模型描述了一个直观的信息传播过程,其过程如下。在独立级联模型下,网络中的个体会被其邻居所影响,这些影响之间是独立的。给定一个种子节点集合$S \subseteq \mathcal{V}$,信息传播在独立级联模型下是如下运作的。定义$S_t$为在$t \geq 0$的时刻激活的节点集合。显然,在$t=0$时刻时,满足$S_0=S$。在$t+1$时刻,每一个在$t$时刻被激活的节点$u \in S_t$会独立地去尝试激活它的出度边指向的未被激活的邻居节点$v \in \mathcal{V} \setminus \cup_{0 \leq i \leq t}S_i$,激活节点$v$的概率等于$p_{u,v}$。当节点$u$尝试了去激活所有它的出度边指向的节点后,它在信息传播的之后过程中将不再会有机会去激活。即在$t$时刻被激活的节点$u$只会在$t+1$时刻去尝试激活它的出度边指向的邻居节点。当$t$满足$S_t = \emptyset$时,整个信息传播过程停止。

\begin{figure}[ht]
   \begin{minipage}{0.48\textwidth}
     \centering
     \includegraphics[width=0.8\linewidth]{figures/tinyGraph.pdf}
     \caption{社交网络中的信息传播概率图$\mathcal{G}$}\label{fig:tinyGraph}
   \end{minipage}
   \hfill
   \begin {minipage}{0.48\textwidth}
     \centering
     \includegraphics[width=0.8\linewidth]{figures/tinyRandomGraph.pdf}
     \caption{随机实例图$g$}\label{fig:tinyRandomGraph}
   \end{minipage}
\end{figure}

以图\ref{fig:tinyGraph}中的社交网络$\mathcal{G}$为例,考虑种子节点集合$S=\left\{v_1\right\}$的信息传播过程。图\ref{fig:tinyGraph}中边上的数值代表着节点之间的传播概率$p^\mathcal{G}_{u,v}$。整个信息传播的过程可以描述如下。在$t=0$时刻,由于节点$v_1$是种子节点集合中唯一的节点,因此节点$v_1$被激活。在$t=1$时刻,因为节点$v_1$在$t=0$时刻被激活,而且图$\mathcal{G}$中存在一条从$v_1$到$v_2$概率为1的边,节点$v_1$将依概率1去激活节点$v_2$。因此,节点$v_2$将在$t=1$时刻被激活,$S_1=\left\{v_2\right\}$。此后,在$t=2$时刻,节点$v_2$将尝试去激活节点$v_3$、$v_4$、$v_5$以及$v_6$。假设这一次信息传播过程如图\ref{fig:tinyRandomGraph}所示,节点$v_3$和$v_5$被激活,则$S_2=\left\{v_3, v_5\right\}$。然后,在$t=3$时刻,由于被激活的节点没有后继节点可被激活,整个信息传播过程在此时刻停止。定义$I\left(S\right)$为在种子节点集合是$S$的条件下,整个信息传播过程中激活的节点数目,代表信息传播过程中的影响力。在上述的图$\mathcal{G}$的一次信息传播过程中,影响力$I\left(S\right)=4$。

给定一个种子节点集合$S$,定义$\mathbb{E}_\mathcal{G}\left[I\left(S\right)\right]$表示种子节点集合在图$\mathcal{G}$中影响力的期望,它等于以种子节点集合为传播源,在图$\mathcal{G}$中信息传播结束时激活节点数目的期望值。在独立级联模型下,影响力最大化问题的目标是寻找一个大小至多为$k$的种子节点集合,使得影响力函数的期望值$\mathbb{E}_\mathcal{G}\left[I\left(S\right)\right]$最大化。给定一个输入$k$,影响力最大化问题可以形式化为如下,

\begin{equation}\label{eq:imProblem}
    \begin{split}
        &S^{\ast} = \arg\max{\mathbb{E}_\mathcal{G}\left[I\left(S\right)\right]}\\
        &s.t.~~S \subseteq \mathcal{V},\left\vert{S}\right\vert = k
    \end{split}
\end{equation}

以图\ref{fig:tinyGraph}的社交网络$\mathcal{G}$为例,给定输入$k=1$来考虑影响力最大化问题。为了解决例子中的影响力最大化问题,根据公式\ref{eq:imProblem}所示,我们需要计算所有$k=1$的种子节点集合的影响力期望值,即每个节点的影响力期望值。对于节点$v_1$组成的种子节点集合,其影响力期望值$\mathbb{E}_\mathcal{G}\left[I\left(\{v_1\}\right)\right]=1+1+4\times0.8=5.2$。对于种子节点集合$\{v_2\}$,影响力期望值$\mathbb{E}_\mathcal{G}\left[I\left(\{v_2\}\right)\right]=1+4\times0.8=4.2$。对于其他的种子节点集合,$\mathbb{E}_\mathcal{G}\left[I\left(\{v_3\}\right)\right]=
\mathbb{E}_\mathcal{G}\left[I\left(\{v_4\}\right)\right]=
\mathbb{E}_\mathcal{G}\left[I\left(\{v_5\}\right)\right]=
\mathbb{E}_\mathcal{G}\left[I\left(\{v_6\}\right)\right]=1$。由此,我们可以得出,在给定图$\mathcal{G}$以及$k=1$的条件下,影响力最大化问题的最优解为$S^\ast=\left\{v_1\right\}$。

在上述的例子中,因为图$\mathcal{G}$的结构简单,并且种子节点集合的大小$k=1$,所以能够直接计算得出种子节点集合的影响力期望值,从而选择得出最优解。而在实际情况下,图$\mathcal{G}$的结构往往会复杂得多,而且$k>1$,很难直接计算种子节点集合的影响力期望值。Kempe等人\upcite{kempe2003maximizing}首先证明了在独立级联模型下,影响力最大化问题是一个NP-hard问题。因此,直接计算出有影响力的节点是十分困难的。为了解决此问题,Kempe等人又证明了在独立级联模型下,影响力函数的期望$\mathbb{E}_\mathcal{G}\left[I\left(S\right)\right]$是单调的以及子模的。这两个性质为求解影响力最大化问题的近似算法提供了理论保证。形式上,一个单调的函数对于任意的节点$u$和任意集合$S$,都满足$f\left(S\cup\left\{u\right\}\right) \geq f\left(S\right)$。而一个子模的函数对于任意的节点$u$以及任意的两个集合$S \subseteq W$,都满足$f\left(S\cup\left\{u\right\}\right) - f\left(S\right) \geq f\left(W\cup\left\{u\right\}\right) - f\left(W\right)$。针对具有单调性以及子模性的函数,Nemhauser等人\upcite{nemhauser1978analysis}提出了一种朴素的贪心算法来解决此类问题。算法的核心思想是首先从一个空的种子节点集合$S=\emptyset$开始,重复地选择当前边际收益(即$f\left(S\cup\left\{u\right\}\right) - f\left(S\right)$)最大的节点$u$加入到种子节点集合$S$中,直到满足种子节点集合的大小为$k$时结束迭代。选择节点$u$的准则可以形式化如下,

\begin{equation}\label{eq:greedyIM}
    u=\arg\max\limits_{w \in V \setminus S}\left({\mathbb{E}_\mathcal{G}\left[I\left(S\cup w\right)\right]-\mathbb{E}_\mathcal{G}\left[I\left(S\right)\right]}\right)
\end{equation}

Nemhauser等人证明了朴素的贪心算法得出解以$1-1/\mathsf{e}$的因子近似于最优解,即任意一个由贪心算法的出来的解$S$都满足$I\left(S\right) \geq (1-1/\mathsf{e}) I\left(S^{\ast}\right)$,其中$S^{\ast}$代表最优解,$\mathsf{e}$为自然常数。虽然贪心算法的核心思想比较简单,但是由于计算影响力期望值的过程是\#P-hard\upcite{Chen2010Scalable}的问题,因此实现该算法并不是简单的。为了解决该问题,Kempe等人\upcite{kempe2003maximizing}提出了使用\textbf{蒙特卡罗方法}(\textit{Monte Carlo method})来对$\mathbb{E}_\mathcal{G}\left[I\left(S\right)\right]$在一定精度内进行估计。蒙特卡罗方法的步骤如下。假设我们对图$\mathcal{G}=\left(\mathcal{V},\mathcal{E}\right)$中的所有的边$e\in\mathcal{E}$都进行抛硬币实验,图$\mathcal{G}$中边的连接的概率为$p\left(e\right)$,我们以$1-p\left(e\right)$的概率移除掉边$e$。定义$g$为得到的结果图,$R_g\left(S\right)$为在图$g$中从种子节点集合$S$出发可达的节点集合。我们需要注意的是,图$g$中的边不再是概率性连接的边,而是确定性的边。对于任意节点$v\in g$,如果在图$g$中存在一条路径从节点集合$S$出发到达节点$v$,那么称节点$v$是从节点集合$S$可达的。Kempe等人\upcite{kempe2003maximizing}证明了$R_g\left(S\right)$的期望值与$\mathbb{E}_\mathcal{G}\left[I\left(S\right)\right]$是相等的,即可表示为如下,

\begin{equation}\label{eq:influenceSpread}
    \mathbb{E}_\mathcal{G}\left[I\left(S\right)\right] = \mathbb{E}_{g\sim\mathcal{G}}\left[I_g\left(S\right)\right]
\end{equation}
其中$I_g\left(S\right) = \left\vert{R_g\left(S\right)}\right\vert$,即在图$g$中的影响力等于从种子节点集合出发可达的节点数目。因此,我们可以通过估计$R_g\left(S\right)$的期望值来估计$\mathbb{E}_\mathcal{G}\left[I\left(S\right)\right]$,即通过估计图$g$中可达的节点数目的期望来估计原图$\mathcal{G}$中的影响力期望值。在实际操作中,我们首先根据原来的社交网络生成多个实例$g\sim\mathcal{G}$,然后对每一个实例进行计算其影响力$I_g\left(S\right)$,最终计算其平均值作为$\mathbb{E}_\mathcal{G}\left[I\left(S\right)\right]$的一个估计。假设我们在估计$\mathbb{E}_\mathcal{G}\left[I\left(S\right)\right]$的过程中生成了$r$个实例图$g$,并且$r$足够大,那么在独立级联模型下,贪心算法能够得到一个$\left(1-1/\mathsf{e}-\varepsilon\right)$的近似最优解,其中$\varepsilon$是一个与图$\mathcal{G}$和$r$相关的常数\upcite{borgs2014maximizing,kempe2005influential}。一般来说,Kempe等人建议设置$r=10,000$,许多其他的工作都采用了相似的设置参数。

尽管朴素的贪心算法是有效的,但是在复杂网络的应用中,算法的效率是极其低的。算法的时间复杂度为$O\left(knmr\right)$。确切来说,算法进行了$k$次迭代来选择种子节点集合,每一次迭代需要对$O\left(n\right)$个节点进行影响力的期望值的估计。每一次估计需要对生成的$r$个实例进行计算,而每一次计算需要消耗$O\left(m\right)$的时间。因此,整个计算过程的时间复杂度为$O\left(knmr\right)$。

对于具有子模性的函数,\textbf{惰性计算}(\textit{lazy evaluations})技术是一种比较知名的优化方法,它能够大大的降低计算的次数,而不改变贪心算法的输出。这个技术首先由Minoux\upcite{minoux1978accelerated}作为一种加速的贪心算法提出,Leskovec\upcite{leskovec2007cost}等人通过实验验证了惰性计算针对影响力最大化问题能够加速近700倍。

即使惰性计算能够提高计算的性能,但是贪心算法仍然在效率上是不足的。究其原因,贪心算法的弊端主要是在于计算影响力期望值的过程中,它需要对$O\left(kn\right)$个节点进行估计。然而,其中大多数估计都是无用的,因为我们只关心影响力期望值最大的节点。在朴素贪心算法的框架下,这些无用的计算又是不可避免的。

为了解决朴素贪心算法的这一弊端,Borgs等人\upcite{borgs2014maximizing}提出了一种新的方法,突破了朴素贪心算法的限制。Tang等人\upcite{tang2014influence}将这种方法称之为\textbf{反向传播采样}(\textit{Reverse Influence Sampling}),并且阐述了其工作原理。为了解释反向传播采样算法的工作原理,我们首先引入如下的概念。

\begin{mydef}[反向可达集合]\label{def:rrSet}
给定一个图$\mathcal{G}=\left(\mathcal{V}, \mathcal{E}\right)$,我们对图中的$\mathcal{G}$中的每一条边$e \in \mathcal{E}$进行抛硬币实验,依概率$1-p\left(e\right)$移除掉边$e$。定义$g$为得到的图,对于任意的节点$v \in \mathcal{V}$,节点$v$在图$g$中的反向可达集合${RR}\left(v\right)$定义为图$g$中可达节点$v$的节点集合。这就是说,如果节点$u \in {RR}\left(v\right)$,则至少在图$g$中存在一条路经从节点$u$到达节点$v$。
\end{mydef}

根据定义\ref{def:rrSet}可知,如果节点$u$在节点$v$的反向可达集合${RR}\left(v\right)$中,那么节点$u$在图$\mathcal{G}$中能够依一定概率通过一条路径到达节点$v$。这也就表示,如果采用节点$u$作为种子节点集合$S=\{u\}$在图$\mathcal{G}$中进行信息传播,那么节点$u$是有一定概率激活节点$v$的。此外,Borgs等人给出了反向可达集合的性质如下,

\begin{mylem}\label{lem:rrSet}
如果一个节点$v$的反向可达集合${RR}\left(v\right)$有$\rho$的概率与节点集合$S$存在交集,那么如果以$S$为种子节点集合在图$\mathcal{G}$中进行信息传播,则节点$v$将依概率$\rho$被激活。
\end{mylem}

\begin{proof}
假定$g$为基于$\mathcal{G}$依概率$1-p\left(e\right)$移除每一条边$e \in \mathcal{E}$生成的实例图。定义$\rho_2$为节点集合$S$在图$g$中可达节点$v$的概率,$\rho_1$为图$g$中存在一条从节点集合$S$到达节点$v$的概率。那么,根据定义\ref{def:rrSet}可知,$\rho_1 = \rho_2$。
\end{proof}

基于以上的理论,反向传播采样方法的算法流程按照如下的两步进行。第一步,首先在图$\mathcal{G}$中等概率地任意选择一个节点$v \in \mathcal{V}$,按照定义\ref{def:rrSet}来生成反向可达集合${RR}\left(v\right)$。然后,重复上述的过程来生成多个实例。第二步,选择$k$个节点来覆盖最多的反向可达集合。当且仅当一个节点$u\in RR\left(v\right)$时,我们称节点$u$覆盖集合$RR\left(v\right)$。最后,方法采用朴素贪心算法来得出一个下限为$1-1/\mathsf{e}$的近似最优解$S$作为结果返回。

反向传播采样方法的核心思想可以描述如下。如果一个节点$u$出现在许多的反向可达集合$RR\left(v\right)$中,那么节点$u$就有更高的概率去激活更多的节点。而且,节点$v$是等概率地从节点集合$\mathcal{V}$中抽取,因此节点$u$将有概率激活图$\mathcal{G}$中更多的节点,即影响力$I\left(\{u\}\right)$的值会更大。这也就是说,如果一个种子节点集合$S$覆盖了最多的反向可达集合$RR\left(v\right)$,那么$S$在图$\mathcal{G}$中将有最大的影响力期望值。

以图\ref{fig:tinyGraph}中的社交网络$\mathcal{G}$为例,考虑在独立级联模型以及$k=1$的条件下,反向传播采样方法的工作流程。第一步,首先反向传播采样方法将等概率地随机从图$\mathcal{G}$中选取节点$v$,然后依照概率$1-p\left(e\right)$移除每一条边$e$得到图$g$,然后计算反向可达集合$RR\left(v\right)$。假设我们选择了节点$v_3$并且得到的结果图$g$如图\ref{fig:tinyRandomGraph}所示,则我们可以计算得到${RR}_1=\left\{v_1, v_2, v_3\right\}$,其中下标表示生成的反向可达集合的序号。这是因为在图$g$中,$v_1$,$v_2$以及$v_3$是可达节点$v_3$的节点。然后,重复上述过程生成多个反向可达集合。假设这个过程中生成的其他反向可达集合为${RR}_2=\left\{v_1\right\}$,${RR}_3=\left\{v_1, v_2\right\}$, ${RR}_4=\left\{v_4\right\}$,${RR}_5=\left\{v_1, v_2, v_5\right\}$以及 ${RR}_6=\left\{v_6\right\}$。在这个情况下,我们可以得出节点$v_1$覆盖了最多的反向可达集合,因为节点$v_1$包含在集合${RR}_1$,${RR}_2$,${RR}_3$,${RR}_5$中。因此,反向传播采样方法返回$S=\left\{v_1\right\}$作为最终结果。

与朴素贪心算法对比,反向传播采样算法之所以效率更高是因为避免了在计算$O\left(kn\right)$次迭代的影响力期望值的无效计算。算法的核心关键点是以反向可达集合$RR$取代了对信息传播的迭代模拟。为了平衡反向传播采样方法的有效性和高效性,算法需要控制生成反向可达集合的数目。Borgs等人证明了,为了在独立级联模型下得到一个$\left(1-1/\mathsf{e}-\varepsilon\right)$的近似最优解,$RR$的数目至少为$\Theta\left(k\left(m+n\right)\log{n}/\varepsilon^3\right)$\upcite{borgs2014maximizing}。

\subsection{传播效率最大化问题}
\label{3subsec:efficiency}

\section{方法描述}
\label{3sec:method}

\section{实验分析}
\label{3sec:experiment}

\section{本章小结}
\label{3sec:conclusion}

