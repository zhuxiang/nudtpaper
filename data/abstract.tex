\begin{cabstract}
近年来,随着互联网技术的发展,在线社交网络已经融入到了人们生活的各方面之中。在线社交网络是现实世界中个体集合和个体之间的连接关系所构成的网络在虚拟网络上的映射,即在线社交网络是真实世界在虚拟网络上的体现与扩展。真实世界中的事件会以信息的形式在社交网络中存在,随着用户之间的社交互动传播与演化,将反作用于现实世界,影响人们的日常行为。在线社交网络拉近了人与人之间的距离,加快了信息传播的速率,丰富了群体的结构关系。因此,在线社交网络成为了社会学、传播学、计算机科学以及系统科学等领域的研究热点。

在线社交网络信息传播分析与自然语言处理、数据挖掘、机器学习、传播学、社会学等学科有着紧密的联系。社交网络中的数据种类繁多、关系结构复杂、信息传播迅速,这些特点使得社交网络中的信息传播分析与传统的信息传播分析有所不同,带来了新的机遇与挑战。本文在已有的研究基础之上,针对社交网络中影响力最大化、实时个性化搜索、文本分类、传播模型学习等问题进行了研究,主要的研究成果如下:
\begin{enumerate}
	\item 在影响力最大化问题方面,本文提出了传播效率最大化问题,并且分析了该问题的复杂性,最后提出了一种反向效率采样算法来解决该问题。传统的影响力最大化问题仅仅考虑了最大化最终的传播范围,而没有考虑节点被激活的时刻,即传播时延。而在实际应用中,如何尽快地使得节点被激活也是十分有意义的。针对上述不足,本研究将传播时延进行了考虑,定义了传播效率函数,提出了传播效率最大化问题。随后,我们证明了该问题是一个NP-难的问题,而且在独立级联模型下计算传播效率是一个\#P-难的问题。我们对传播效率函数进行分析,证明传播效率函数在独立级联模型下是具有子模性的。最后,本研究设计了一种反向传播采样算法来解决该问题。
	\item 在实时个性化搜索方面,本文提出了一种结合语义扩展和质量模型的实时个性化搜索框架。在传统的搜索模式中,往往是用户输入关键词,系统进行检索后返回排序后的结果。而在社交网络中,信息产生速率快,信息以数据流的形式出现,同时,用户希望系统自动地根据用户偏好推送信息。针对这些不足,本研究提出了一种基于语义扩展和文本质量的实时个性化搜索框架,该框架综合考虑了用户的偏好、语义特征和社交属性。我们采用了一种布尔逻辑关键词过滤(Boolean Logic Keyword Filter)的用户模型。该模型依靠外部搜索引擎提供的知识进行建立,建立的用户模型充分利用了查询扩展以及检索结果的重排序来提高推荐结果的相关性。同时我们还使用了一种基于逻辑回归的文本质量模型,该模型利用推文的社交属性来评估其文本的质量,使得返回结果中的文本包含更有用的信息。
	\item 在文本分类方面,本文提出了一种基于文本摘要和卷积神经网络的文本分类方法。社交网络中的文本数据量大,话题种类较多,传统的词袋模型面临着维度爆炸问题。同时社交网络中的文本包含着丰富的上下文关系,传统的分类方法都没有对这个信息进行利用。针对上述不足,我们提出了一种结合词向量模型和卷积神经网络的文本分类算法,该算法控制了文本表示的维度,保留了上下文关系的局部特征。算法包含了文本摘要的提取、同时利用外部语料库训练结果进行词语的向量化以及卷积神经网络。实验证明了算法的正确性和有效性。
	\item 在传播模型学习方面,本文提出了一种基于概率阅读的事件传播模型。在实际应用中,一个事件往往由多条信息组成,因此研究事件的传播分析需要考虑多个传播网络融合的问题。其次,社交网络中的垃圾用户对传播影响范围的计算造成了偏差。因此,需要训练模型对垃圾用户进行过滤,去除这些噪声。最后,信息推送至用户,用户阅读到信息是一个概率性的事件,需要建立模型来计算用户阅读到信息的概率。针对上述问题,我们提出了基于概率阅读的事件传播模型,模型对上述的三个问题进行了综合考虑,进行训练后得到的模型能够真实的反应出事件的传播范围。
\end{enumerate}

综上所述,本文对在线社交网络中的影响力最大化问题、实时个性化搜索、文本分类以及传播模型等关键技术进行了研究,在真实数据集上验证了方法的有效性,对社交网络信息传播提供了理论依据和实际应用方法。
\end{cabstract}
\ckeywords{社交网络; 影响力最大化; 实时个性化搜索; 文本分类; 传播模型}

\begin{eabstract}
Recently, with the development of Internet technology, the online social network plays an important role in people's daily life. The online social network is the representation of real world network which consists of the individuals and the relationships among individuals. It means online social network is the reflection and extension of real world. The information about the event which happens in the real world will exist in online social network. And it will propagate and evolve with the interactions among the users in online social network. And what's more, it will react on the real world and influence people's behaviors and actions. The online social network narrows the gap between the individuals, accelerates the propagation speed and fertilizes the relationships among individuals. So the online social network becomes one of the research highlights in the field of sociology, communication, computer science, system science, and etc.

The analysis of information diffusion in online social network is related to natural language processing, data mining, machine learning, communication, sociology, and etc. In online social network, there are many types of data, complex relationships and fast propagating information. Those characteristics make it different for information diffusion analysis in online social network from the traditional analysis. It brings new challenges as well as new opportunities. This paper aims at influence maximization, real-time personalized search, text classification and information diffusion learning based on the existing related work. The main achievements are as follows:

\begin{enumerate}
	\item In terms of influence maximization problem, this paper proposes the influence efficiency maximization problem, analyses the complexity of that problem and design a reverse efficiency sampling algorithm to solve that problem. Traditional influence maximization problem only take the final influence into consideration, neglecting the time stamp that node is activated, which is called propagation time delay. While in practical application, it is meaningful to activate nodes as soon as possible. To solve the problem, we take propagation time delay into account, define the influence efficiency function and propose the influence efficiency maximization problem. After that, we prove that problem is NP-hard under independent cascade model. And what's more, the computation of influence efficiency is \#P-hard under independent cascade model. We also prove that the influence efficiency function is submodular. At last, we design a reverse efficiency sampling algorithm to solve the problem.
	\item In the terms of real-time personalized searching, this paper propose a framework for real-time personalized searching which integrates semantic extension and quality model. In traditional information retrieval model, the system returns the re-ranked results based on user input. While in online social network, the information generates in high speed and emerges as data stream. Moreover, it is proper to push information automatically according to user's intention. To solve those problems, we propose a framework for real-time personalized searching that integrates semantic extension and quality model. We adopt a user model based on boolean logic keyword filter. It constructs based on external knowledge base, leverage the query extension to re-rank the results according to the relevance. Furthermore, we utilize a text quality model based on logistic regression. It evaluates text's quality based on the social attributes, making the retrieval text meaningful.
	\item In the terms of text classification, this paper proposes a text classification algorithm based on text summarization and convolutional neural network. In online social network, there are enormous text data and a large variety of topics. So traditional bag-of-word model confronts the problem of dimension explosion. Besides, there is context-dependent in the text. While traditional methods make no use of that property. To solve that problem, we propose a text classification algorithm integrating word vector model and convolutional neural network. The algorithm fixes the dimensionality and keep the local feature of the context. It includes text summary extraction, text vectorization and convolutional neural network. Finally, the experiments verify the effectiveness of the algorithm.
	\item In the terms of diffusion model learning, this paper proposes a diffusion model based on probabilistic reading. In practice, an event consists of many information. So we need to take multiple network convergence into consideration. And furthermore, the spam users in online social network bring noise into the evaluation of influence. So we need to construct a spam user filter to get rid of the noise. Moreover, whether the user receives the information or not is probabilistic when the information is pushed to user. So we need to model user's reading probability. To solve those problems, we propose a diffusion model based on probabilistic reading, which considers those problems comprehensively. After training, the model can reflect the actual influence of an event.
\end{enumerate}

In a conclusion, this paper aims at influence maximization problem, real-time personalized searching, text classification and diffusion model learning in online social network. We carry out experiments on real world dataset, and the experimental results show the effectiveness of our methods. The methods for information diffusion analysis in online social network are practical.
\end{eabstract}
\ekeywords{Social Network, Influence Maximization, Real-time Personalized Search, Text Classification, Diffusion Model}

