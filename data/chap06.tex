\chapter{总结与展望}
本章首先对全文的工作内容进行总结,概括了研究工作的主要创新点,最后对本领域内未来的研究工作进行展望。
\section{本文工作总结}
社交网络是真实世界在虚拟网络上的一个映射,反映了真实世界中的各个事件、群体舆论以及相互关系。现实世界中讨论的内容、日常交互行为以及突发事件爆发产生的信息都将在社交网络中传播,并且基于社交网络的网络结构以及交互行为在短时间内飞速传播和演化,反作用于现实社会,影响真实世界中人们的观点和行为。因此,对社交网络中信息传播的研究有助于理解信息在虚拟网络和真实世界中的传播机理、促进真实信息的良好传播和抑制有害信息的传播、从海量的信息中挖掘对用户有价值的信息等等。本文主要从影响力最大化、数据流实时推送、社交网络文本分类和传播模型学习四个方面研究社交网络信息传播的关键技术,其主要研究内容和研究成果如下:
\begin{enumerate}
	\item 影响力最大化:社交网络中传播效率最大化问题。传统的影响力最大化问题只考虑了最大化传播影响的范围,没有考虑节点被激活的时刻,即传播时延。这可能导致用户不能及时的接收到对其有用的信息。针对这一不足,本文将传播时延考虑在内,提出了一个新的问题,即传播效率最大化问题,研究如何选择种子节点集合使得传播效率最大化。我们分析了该问题的复杂性,证明该问题是一个NP-难问题,而且在独立级联模型下计算传播效率的过程是一个\#P-难的问题。同时,我们分析了该问题的性质,证明了传播效率函数在独立级联模型下是具有子模性(\textit{submodular})的。然后由浅入深设计了三个算法来解决该问题,通过真实数据验证了算法的有效性。
	\item 数据流实时推送:基于语义扩展和文本质量的实时个性化搜索。传统的信息检索模式往往是用户输入关键词,系统进行检索后返回排序后的结果。社交网络的信息产生速率快,信息以数据流的形式进行传播,同时用户希望系统能够自动地推送其感兴趣的信息。因此,社交网络中的实时个性化搜索与传统的信息检索不尽相同。针对该问题,本文提出了一种基于语义扩展和文本质量的实时个性化搜索框架,该框架综合考虑了用户的偏好、语义特征和社交属性。该框架使用了外部知识库来进行查询扩展,更加准确的理解用户的意图,同时建立一个信息的质量模型,利用推文的社交属性来评估其文本的质量,使得返回结果中的文本包含更有用的信息。
	\item 社交网络文本分类:基于卷积神经网络的文本分类研究。社交网络中的文本数据量大,话题种类较多,对文本分类突出了新的要求。社交网络中的信息量随着时间而增加,话题种类随着增加,传统的词袋模型面临着维度爆炸问题。同时社交网络中的文本包含着丰富的上下文关系,传统的分类方法都没有对这个信息进行利用。针对上述不足,我们提出了一种结合词向量模型和卷积神经网络的文本分类算法,该算法控制了文本表示的维度,保留了上下文关系的局部特征。算法包含了文本摘要的提取、同时利用外部语料库训练结果进行词语的向量化以及卷积神经网络。
	\item 传播模型学习:基于概率阅读的事件传播模型研究。在实际应用中,一个事件往往由多条主要的信息组成,因此一个事件的传播将由多个传播网络组合而成。因此衡量事件的影响范围需要考虑多个传播网络的融合问题。其次,社交网络中存在很多的垃圾用户,这些用户活跃度低,或者是由水军控制,会影响到影响力量化的计算,因此需要建立垃圾用户过滤的机制,除去这些干扰噪声。最后,信息推送至用户,用户阅读到信息是一个概率性的事件,需要建立模型来计算用户阅读到信息的概率。针对上述问题,我们提出了基于概率阅读的事件传播模型,模型对上述的三个问题进行了综合考虑,进行训练后得到的模型能够真实的反应出事件的传播范围。
\end{enumerate}
\section{未来工作展望}
在线社交网络信息传播的研究是一个富有挑战性的问题。社交网络是基于信息系统以及互联网技术的发展而形成的应用,但是其中的信息产生、传播和演化,用户群体之间的交互与博弈,网络结构的形成与演化涉及了众多的学科领域,包含了社会学、传播学、心理学、系统科学、应用数学等等学科。本文的研究仅仅在社交网络信息传播相关的技术研究进行了小小的探索,虽然取得了一些阶段性的成果,但是仍然只是窥见了在线社交网络信息传播研究的冰山一角。该问题依旧存在很多亟待研究的方面,结合自身博士阶段的工作,未来的研究工作可以围绕如下几个方面展开:
\begin{enumerate}
	\item 影响力最大化问题。寻找到社交网络中有影响力的个体或者群体是信息传播的基础,该问题能够在一定约束条件下,寻找到传播信息效率最快的途径。在社交网络中存在着多种多样的约束条件,不同的条件使得问题具有不同的性质和形式。社交网络中也存在这博弈竞争的情况,在竞争的情况下如何将正确的信息有效地传播也是研究的一个方面。影响力最大化的解决方法分为利用问题子模性设计的有理论保证的算法和利用网络性质设计的启发式算法。在实际应用中,如何使得性能和效率达到一个良好的效果也是应用的关键问题。
	\item 数据流实时推送。社交网络中的数据流实时个性化推送包含了对用户意图理解以及信息语义分析两方面。由于社交网络中的短文本性质,目前向量空间模型是最流行的模型,但是训练方法和途径多种多样。如何建立语言模型,将自然语言转换到可计算的向量空间是众多语义分析的核心内容。同时社交网络中的信息产生速率快、话题种类繁多,信息中包含的有意义的内容层次不齐,建立一个信息的质量评估模型是过滤掉大量无用信息的一个途径。用户接受信息的数量是有限的,如何平衡信息推送的及时性和数量的矛盾是一个实际问题。
	\item 社交网络文本分类。随着深度学习技术的发展,自然语言处理领域又得到了一个文本分析的探索方向。神经网络的结构使得文本信息的上下文关系得到了良好的应用,局部特征能够自动地被网络所获取。神经网络用于训练词向量模型是深度学习在语言模型上的一个应用,如何得到更准确的向量模型对于后续的文本分析都有着重要的意义。词向量作为文本分类的输入,如何构建良好的神经网络对于分类效果有着关键性的影响。同时,在深层的网络训练过程中,如何使得网络能够快速的收敛也是一个实际的问题。
	\item 传播模型学习。随着社交网络的发展,对于信息传播影响的评估成为了另一个研究的热点。以事件为粒度来分析信息的传播影响范围涉及到众多的问题,如何自动的获取代表该事件的信息,并且融合多个信息传播网络是一个应用型的问题。社交网络中的用户数目庞大,但是其中也不乏许多水军、机器人账户,这些用户在表面上能够造成传播的假象。因此,如何自动的识别这些账户,在传播的评估中去除这些用户的影响,还原一个真实的传播影响范围是一个研究热点。用户评论和转发是一个显式的行为,但是阅读行为则是一个隐式的行为,如何更具用户的行为建模,判断用户是否阅读到信息是另一个问题。
\end{enumerate}