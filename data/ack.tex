
%%% Local Variables:
%%% mode: latex
%%% TeX-master: "../main"
%%% End:

\begin{ack}
求学生涯,已逾十载。蓦然回首,感慨万千。二零零六年,怀揣着梦想,背上行囊,踏上了北上的列车。京城四年的大学生活让我领略了北方的豪迈,“自强不息,厚德载物”的训诲犹在耳旁,“水清木华,荷塘月色”的美景犹映眼帘。二零一零年,大学毕业,携笔从戎,南下至千年古城长沙,来到庄严肃穆的国防科学技术大学,“厚德博学,强军兴国”的口号振聋发聩,“兢兢业业,无悔青春”的态度令人肃然起敬。如今博士学业即将结束,离开的日子也越来越近,所以伏案疾笔,将五年的所学所得都书写于此。虽然本人天资鲁钝,但自认为勤勉,不求妙笔生花,但使字句通顺,言之有理,书生意气,足以慰藉平生。

初入计算机学院之时,拜师于贾焰教授门下。贾教授治学严谨,神思敏捷。虽然平日繁忙,但是事情轻重缓急,井然有序,不差毫厘。七年的细心教诲,毕生技艺,毫无保留地倾力相授。如今毕业在即,授业之恩,不敢稍忘。如若他日有所成,必定登门拜访致谢。

周斌研究员,师门的指导老师,能够得到他的指导,学业事半功倍。周老师常常秉烛伏案,阅读文献,指导学生的研究工作,直到深夜才离开。每每看到周老师的工作态度,方才明白聚沙成塔,水滴石穿的道理。仿效周老师的学习态度,激励自己,坚持下来,方有小成。对比周老师的严谨态度,在求期间取得的成果甚少,每思至此,唯叹才疏学浅,无以为报。

李爱平研究员,能够得到李老师的指导是很幸运的。李老师常常教诲学生,为学先为人的道理,通晓这个道理,对学习生活都大有裨益。李老师为人亦师亦友,或传道授业解惑与室屋之内,或协力拼搏于绿茵赛场。期望他日再会,把酒言欢,好不痛快。

裴健教授,数据挖掘领域的大师。二零一五年,有幸得到机会远赴加拿大交流学习。交流学习时间虽然只有三月有余,但是在裴教授高瞻远瞩的指导下,对我的眼界大有开阔。

师门之下,门庭若市,或温和敦厚,或能言善辩,都是忠厚老实的人,在需要帮助的时候,都能施之援手。教研室里吴泉源教授、杨树强研究员、韩伟红研究员、韩毅老师、黄九鸣老师、甘亮老师、江荣老师等人的指导和关照让我更快的成长。同级的聂原平、徐菁等人,共同进退,迎难而上。同组的朱俊星、全拥、邓璐、刘强等师弟师妹们一同协作,攻克难题让我的科研生活并不孤单。

感谢研究生期间的室友朱孟斌以及国外的室友张盛东、张灵康在生活学习上的互帮互助。感谢裴健教授IDEAL实验室的杨禹、毛翔博、高传聪、胡菊花、丛梓存以及一同交流访问的王喆锋、雷鸣涛、张东翔等人,异国他乡是你们给我家乡一般的温暖。筵席终有散,天人各一方。愿有佳期,欢聚一堂,挥斥方遒,快意人生。

最后,感谢我的父母和妻子,你们是我一生最大的财富。
\end{ack}
