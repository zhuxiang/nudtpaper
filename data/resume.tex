\begin{resume}

  \section*{发表的学术论文} % 发表的和录用的合在一起

  \begin{enumerate}[{[}1{]}]
  \addtolength{\itemsep}{-.36\baselineskip}%缩小条目之间的间距,下面类似
  \item Zhu Xiang, Huang Jiuming, Zhou Bin, Li Aiping, Jia Yan. Real-time personalized twitter search based on semantic expansion and quality model. In press. DOI:10.1016/j.neucom.2016.10.082. (已被Neurocomputing录用. SCI 源刊.)
  \item Zhu Xiang, Nie Yuanping, Li Aiping. Demographic Prediction of Online Social Network Based on Epidemic Model. Asia-Pacific Web Conference Workshops, 2014, 93-103. (EI收录,检索号:20143518106890)
  \item Zhu Xiang, Nie Yuanping, Jin Songchang, Li Aiping, Jia Yan. Spammer detection on online social networks based on logistic regression. Web-Age Information Management Conference Workshops, 2015, 29-40. (EI收录,检索号:20155201713855)
  \item Zhu Xiang, Jia Yan, Nie Yuanping, Qu Ming. Event Propagation Analysis on Microblog. Journal of Computer Research and Development, 2015, 52(2): 437-444. (EI收录,检索号:20151400720664)
  \item Zhu Xiang, Huang Jiuming, Zhou Bin, Han Yi. Chinese Article Classification Oriented to Social Network Based on Convolutional Neural Networks. IEEE International Conference on Data Science in Cyberspace, 2016, 33-36. (EI收录,检索号:20171403533722)
  \item Zhu Xiang, Huang Jiuming, Zhu Sheng, Chen Ming, Zhang Chenlu, Li Zhenzhen, Huang Dongchuan, Zhao Chengliang, Li Aiping, Jia Yan. NUDTSNA at TREC 2015 Microblog Track: A Live Retrieval System Framework for Social Network based on Semantic Expansion and Quality Model. The Twenty-Fourth Text REtrieval Conference Proceedings, 2015.
  \item Zhang Lumin, Jia Yan, Zhu Xiang, Zhou Bin, Han Yi.  User-Level Sentiment Evolution Analysis in Microblog. China Communications, 2014, 11(12):152-163. (SCI收录,检索号:WOS:000347669800016)
  \item Nie Yuanping, Jia Yan, Li Shudong, Zhu Xiang, Li Aiping, Zhou Bin.  Identifying users across social networks based on dynamic core interests. Neurocomputing, 2016, 210:107-115. DOI:10.1016/j.neucom.2015.10.147. (SCI收录,检索号:WOS:000384866200011)
  \item Jin Songchang, Zhang Yuchao, Nie Yuanping, Zhu Xiang, Yin Hong, Li Aiping, Yang Shuqiang. A Parallel and Scalable Framework for Non-overlapping Community Detection Algorithms. Asia-Pacific Web Conference Workshops, 2014, 115-126. (EI收录,检索号:20143518106892)
  \item Wang Wei, Yang Linlin, Liao Qing, Zhu Xiang, Zhang Qian. TiSA: Time-dependent social network advertising. IEEE International Conference on Communications, 2015, 1188-1193. (EI收录,检索号:20160201792163)
  \end{enumerate}

  \section*{攻读博士学位期间参与的工程项目}
  \begin{enumerate}[{[}1{]}]
  \addtolength{\itemsep}{-.36\baselineskip}%缩小条目之间的间距,下面类似
  \item 社交网络分析与网络信息传播的基础研究,973项目(2013CB329601),主要参与者。
  \item 面向互联网的xxxx信息分析挖掘关键技术研究,国家科技支撑计划课题(2012BAH38B04),主要参与者。
  \item 电子xx关键技术研究,国家科技支撑计划课题(2012BAH38B06),主要参与者。
  \item 虚拟身份管理技术,863项目(2012AA01A401),参与者。
  \item 虚拟资产保全技术,863项目(2012AA01A402),参与者。
  \end{enumerate}
%  \section*{研究成果} % 有就写,没有就删除
%  \begin{enumerate}[{[}1{]}]
%  \addtolength{\itemsep}{-.36\baselineskip}%
%  \item 任天令, 杨轶, 朱一平, 等. 硅基铁电微声学传感器畴极化区域控制和电极连接的
%    方法: 中国, CN1602118A. (中国专利公开号.)
%  \item Ren T L, Yang Y, Zhu Y P, et al. Piezoelectric micro acoustic sensor
%    based on ferroelectric materials: USA, No.11/215, 102. (美国发明专利申请号.)
%  \end{enumerate}
\end{resume}
