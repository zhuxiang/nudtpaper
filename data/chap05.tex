\chapter{基于概率阅读的事件传播模型研究}
\label{chap5:main}
信息传播的\textbf{信息传播模型}分析是社交网络分析中的一个重要研究点,网络热点事件通过社交网络平台迅速传播、发酵,从而在短时间内形成信息爆发、造成极大的影响。社交网络加速了新闻、通知等信息的传播,使得人们接受信息的速率加快,传播信息的成本降低,在一定程度上方便了人们的生活。但是,在社交网络中,人人都可以是信息的生产者、传播者和接收者,每一个社交网络用户都能通过平台发布传播信息。在社交网络中制造舆论热点,进行信息传播的代价相对于传统媒体较低,因此很容易被不法分子利用,对社会安全以及人身财产等造成损失。已有的信息传播模型研究主要包括独立级联模型和线性阈值模型等,这些模型对信息的传播进行了量化,对信息的传播范围进行了传播范围的评估。但是,仍然有一些需求在实际应用中得不到满足。

在信息传播过程中,一个被激活的节点(信息接收者)将尝试继续传播该信息,去激活下一个节点,从而产生级联效应。在实际应用中,一个事件往往由多条主要的信息组成,因此一个事件的传播将由多个传播网络组合而成。所以事件的传播模型首先需要考虑的是多个传播网络的融合问题。其次,社交网络中存在很多的垃圾用户,这些用户活跃度低,或者是由水军控制,会影响到影响力量化的计算,因此需要建立垃圾用户过滤的机制,除去这些干扰噪声。最后,信息的叶子节点,即事件传播的最终的接收者(不再进行下一轮传播)是否阅读到了该信息是概率性的,因此根据用户的社交行为来为用户阅读信息进行建模能够提高影响力量化的准确度。出于精确量化事件传播的影响力的需求,本章提出了一个基于概率阅读的事件传播模型,将多信息传播网络融合、垃圾用户过滤和概率阅读模型综合考虑,为事件在社交网络中的传播进行建模,精确量化事件的传播影响力。

本章的主要工作可以总结如下。首先,基于事件中单条信息的传播网络,我们对多个传播网络进行融合,去除掉重复的节点。其次,利用用户在社交网络中的社交属性,基于逻辑回归模型进行建模,使用梯度下降法进行参数训练,最终得到模型用于垃圾用户过滤。然后,利用用户在社交网络中的时间线上的社交属性,对用户阅读到信息的概率进行建模,对概率阅读模型进行参数训练,得到的模型最终用来预测用户阅读到信息的概率。最后,本章对提出的模型进行了验证,实验使用新浪微博的真实数据集,实验结果证明了模型的有效性。

本章的内容组织如下:第\ref{sec5:motivation}节介绍本章的研究动机,讨论了传统的传播模型的不足之处以及研究基于概率阅读的事件传播模型的意义。第\ref{sec5:definition}节介绍了相关的定义,对本章中所研究的问题进行了定义。第\ref{sec5:method}节介绍了方法描述,详细地阐述了本章提出的模型以及建立过程。第\ref{sec5:experiment}节进行了实验分析,通过实验结果验证了模型的正确性,并对实验结果进行了分析。最后,第\ref{sec5:conclusion}节对本章进行了总结。
\section{研究动机}
\label{sec5:motivation}

\section{相关定义}
\label{sec5:definition}

\section{方法描述}
\label{sec5:method}

\section{实验分析}
\label{sec5:experiment}

\section{本章小结}
\label{sec5:conclusion}